\chapter{Sequences}

% ==================================================================================================

\section{Definition}
A sequence is a specific type of function. A function is a specific type of binary relations. All sequences are functions, and all functions are binary relations, but not all binary relations are functions. Are all functions sequences?
\begin{definition}
  A sequence is a function $f:\N\mapsto\R$. For convenience, in this course we define a sequence to be $f:\N^+\mapsto\R$ instead.
\end{definition}
\begin{notation}
  We can also denote a sequence as $(a_n)_{n\geq 1}$, where $a_n = f(n)$. Writing $a_n$ or $a_1, a_2, ...$ is also fine as long as it is clear enough that they denote a sequence.
\end{notation}

% ==================================================================================================

\section{Monotonicity}
\begin{definition}
  A sequence is increasing $\iff$ $a_{n+1}\geq a_n\forall n\geq 1$.
\end{definition}
\begin{definition}
  A sequence is decreasing $\iff$ $a_{n+1}\leq a_n\forall n\geq 1$.
\end{definition}
\begin{definition}
  A sequence is monotonic $\iff$ it is either increasing or decreasing (or both).
\end{definition}
\begin{definition}
  A sequence is strictly increasing $\iff$ $a_{n+1}>a_n\forall n\geq 1$.
\end{definition}
\begin{definition}
  A sequence is strictly decreasing $\iff$ $a_{n+1}<a_n\forall n\geq 1$.
\end{definition}

% ==================================================================================================

\section{Convergence and divergence}
\begin{definition}
  A sequence converges if it converges to a limit $l \in \R$, $\infty$, or $-\infty$. A sequence diverges if it does not converge.
\end{definition}

% --------------------------------------------------------------------------------------------------

\subsection{$\epsilon - N$ definition of convergence}
Let $(a_n)_{n \geq 1}$ be a sequence.
\begin{definition}
  $(a_n)_{n \geq 1}$ converges to a limit $l$ if and only if
  \[
    \forall \epsilon > 0, \exists N \in \N (\forall n \in \N(n > N \implies \abs{a_n - l} < \epsilon))
  \]
  We denote this as $\lim_{n \to \infty} a_n = l$ or $(a_n)_{n \geq 1} \to l$. As above, $a_n \to l$ would also be fine. 
\end{definition}
Note that $n$ should be strictly greater than $N$.
\begin{intuition}
  If somebody gives us an arbitrary $\epsilon > 0$, then we can always find an $N$ such that every term after that will be within $\epsilon$ of the limit. So if $\epsilon$ is very close to 0, then this guarantees that after some point, all terms in the sequence will be very close to the limit.
\end{intuition}
The statement of the definition itself isn't very helpful when we try to prove limits with it. A general template for answering problems is as follows:
\begin{template}
  Given some $\epsilon > 0$, we want to find limit $l$ and some $N \in \N$ such that $\forall n > N, \abs{a_n - l} < \epsilon$. We speculate that $l = \text{[some value]}$. Now,
  \begin{align*}
    \abs{a_n - l} < \epsilon &\iff \text{[substitute $a_n$ and $l$]} \\
    &\iff \text{[simplify]} \\
    &\iff \text{[remove the absolute operators and justify]} \\
    &\iff ... \\
    &\iff n > \text{[some function of $\epsilon$, let's call this $N(\epsilon)$]} \\
  \end{align*}
  Choose $N \in \N$ such that $N > N(\epsilon)$. Then $\forall n > N, n > N(\epsilon)$.
  \begin{align*}
    \abs{a_n - l} &= ... \\
    &< \epsilon \\
  \end{align*}
  Since $\epsilon$ arbitrary, $\forall \epsilon > 0, \exists N \in \N (\forall n \in \N (n > N \implies \abs{a_n - l} < \epsilon))$, where $l = \text{blah}$. So limit is blah.
\end{template}
If the question gives us the limit and asks us to prove it directly, just change around the sentences a bit.
\begin{eg}
  (MMT 1, 6(a)) Let $\alpha$ be a positive, real constant. Use a direct proof to show that the sequence $(a_n)_{n \geq 1} = (n ^ {-\alpha})$ converges to 0 as $n$ tends to infinity.
  \begin{proof}
    Given some $\epsilon > 0$, we want to find some $N \in \N$ such that $\forall n > N, \abs{a_n - l} < \epsilon$, where $l = 0$. Now,
    \begin{align*}
      \abs{a_n - l} < \epsilon &\iff  \abs{n ^ {-\alpha} - 0} < \epsilon \\
      &\iff \abs{\frac{1}{n ^ \alpha}} < \epsilon \\
      &\iff \frac{1}{n ^ \alpha} < \epsilon && \text{since $n > 0 \implies \frac{1}{n ^ \alpha} > 0$} \\
      &\iff n ^ \alpha > \frac{1}{\epsilon} \\
      &\iff n > \frac{1}{\epsilon ^ \frac{1}{\alpha}} = \epsilon ^ {-\frac{1}{\alpha}}
    \end{align*}
    Choose $N \in \N$ such that $N > \epsilon ^ {-\frac{1}{\alpha}}$. Then $\forall n > N, n > \epsilon ^ {-\frac{1}{\alpha}}$.
    \begin{align*}
      \abs{a_n - l} &= \abs{n ^ {-\alpha} - 0} \\
      &= \frac{1}{n ^ \alpha} && \text{from above} \\
      &< \frac{1}{(\epsilon ^ {-\frac{1}{\alpha}}) ^ \alpha} \\
      &= \epsilon
    \end{align*}
    Since $\epsilon$ arbitrary, $\forall \epsilon > 0, \exists N \in \N (\forall n \in \N (n > N \implies \abs{a_n - l} < \epsilon))$. So limit is 0.
  \end{proof}
\end{eg}

% --------------------------------------------------------------------------------------------------

\subsection{Converging to $\infty$ and $-\infty$}
\begin{definition}
  $(a_n)_{n \geq 1}$ converges to $\infty$ if and only if
  \[
    \forall r \in \R, \exists N \in \N (\forall n \in \N (n \geq N \implies a_n > r))
  \]
\end{definition}
Note that this time, the requirement is $n$ greater than or equal to $N$, but not strictly greater than $N$. I don't know why this is the case, but we'll just accept it. Maybe the slides are wrong.
\begin{intuition}
  If $r$ is arbitrarily large, then this guarantees that after some point, all the terms will be larger than the arbitrarily large $r$, i.e. mega large.
\end{intuition}
\begin{eg}
  Use a direct proof to show that the sequence $(a_n)_{n \geq 1} = n!$ converges to $\infty$.
\end{eg}
\begin{proof}
  Given some $r \in \R$, we want to find some $N \in \N$ such that $\forall n \geq N, a_n > r$.
  \begin{align*}
    a_n > r &\iff n! > r \\
    &\iff n \cdot (n - 1) \cdot (n - 2) \cdot ... \cdot 1 > r
  \end{align*}
  Choose $N \in \N$ such that $N > r$. Then $\forall n \geq N, n > r$.
  \begin{align*}
    a_n &= n! \\
    &= n \cdot (n - 1) \cdot (n - 2) \cdot ... \cdot 1 \\
    &\geq n \cdot 1 \cdot 1 \cdot ... \cdot 1 \\
    &= n \\
    &> r 
  \end{align*}
  Since $r$ arbitrary, $\forall r \in \R, \exists N \in \N (\forall n \in \N (n \geq N \implies a_n > r))$. So $a_n$ converges to $\infty$.
\end{proof}
\begin{definition}
  $(a_n)_{n \geq 1}$ converges to $-\infty$ if and only if
  \[
    \forall r \in \R, \exists N \in \N (\forall n \in \N (n \leq N \implies a_n < r))
  \]
\end{definition}
Also note that $n$ is less than or equal to $N$, but not strictly less than $N$.
\begin{eg}
  Use a direct proof to show that the sequence $(a_n)_{n \geq 1} = \ln \frac{1}{n}$ converges to $-\infty$.
\end{eg}
\begin{proof}
  Given some $r \in \R$, we want to find some $N \in \N$ such that $\forall n \geq N, a_n < r$.
  \begin{align*}
    a_n < r &\iff \ln \frac{1}{n} < r \\
    &\iff \frac{1}{n} < e ^ r \\
    &\iff n > e ^ {-r}
  \end{align*}
  Choose $N \in \N$ such that $N > e ^ {-r}$. Then $\forall n \geq N, n > e ^ {-r}$.
  \begin{align*}
    a_n &= \ln \frac{1}{n} \\
    &< \ln \frac{1}{e ^ {-r}} \\
    &= \ln e ^ r \\
    &= r
  \end{align*}
  Since $r$ arbitrary, $\forall r \in \R, \exists N \in \N (\forall n \in \N (n \geq N \implies a_n < r))$.
  So $a_n$ converges to $-\infty$.
\end{proof}

% --------------------------------------------------------------------------------------------------

\subsection{Divergence}
A sequence diverges if it does not converge to a limit $l \in \R$, does not converge to $\infty$, and does not converge to $-\infty$.
\begin{eg}
  Prove that the sequence $(a_n)_{n \geq 1} = (-1) ^ n$ diverges.
\end{eg}
\begin{proof}
  We will prove that $a_n$ diverges by contradiction.

  Assume $a_n$ converges to a limit $l \in \R$.
  Then 
  \[
    \forall \epsilon > 0, \exists N \in \N (\forall n \in \N (n > N \implies \abs{a_n - l} < \epsilon))
    \]
  Suppose $\epsilon = 1$. Choose arbitrary $n > N$. Then $2n > N$ and $2n + 1 > N$.
  \begin{align*}
    \abs{a_{2n} - l} < \epsilon &\iff \abs{(-1) ^ {2n} - l} < 1 \\
    &\iff \abs{1 - l} < 1 \\
    &\iff -1 < 1 - l < 1 \\
    &\iff -2 < -l < 0 \\
    &\iff 0 < l < 2 \\
    \abs{a_{2n + 1} - 1} < \epsilon &\iff \abs{(-1) ^ {2n + 1} - l} < 1 \\
    &\iff \abs{-1 - l} < 1 \\
    &\iff -1 < -1 - l < 1 \\
    &\iff 0 < -l < 2 \\
    &\iff -2 < l < 0
  \end{align*}
  Since no $l \in \R$ can satisfy both $0 < l < 2$ and $-2 < l < 0$, we have a contradiction.
  So $a_n$ does not converge to a limit $l \in \R$.

  Assume $a_n$ converges to $\infty$.
  Then 
  \[
    \forall r \in \R, \exists N \in \N (\forall n \in \N (n \geq N \implies a_n > r))
    \]
  Suppose $r = 2$. Since $a_n = -1$ or $1$, and $-1 < 2$ and $1 < 2$, we have a contradiction.
  So $a_n$ does not converge to $\infty$.

  Assume $a_n$ converges to $-\infty$.
  Then 
  \[
    \forall r \in \R, \exists N \in \N (\forall n \in \N (n \geq N \implies a_n < r))
    \]
  Suppose $r = -2$. Since $a_n = -1$ or $1$, and $-1 > -2$ and $1 > -2$, we have a contradiction.
  So $a_n$ does not converge to $-\infty$.

  Therefore, $a_n$ diverges.
\end{proof}

% ==================================================================================================

\section{Sandwich theorem (squeeze theorem)}
\begin{definition}
  Suppose $(l_n)_{n \geq 1} \to l$ and $(u_n)_{n \geq 1} \to l$, where $l \in \R$.
  \[
    \exists N \in \N : \forall n \geq N, l_n \leq a_n \leq u_n \implies a_n \to l
  \]
\end{definition}
Again, $n$ is greater than or equal to, but somehow not strictly greater than, $N$.
\begin{proof}
  Given some $\epsilon > 0$, since $l_n \to l$,
  \[
    \exists N_1 \in \N : \forall n > N_1, \abs{l_n - l} < \epsilon
  \]
  Also, since $u_n \to l$,
  \[
    \exists N_2 \in \N : \forall n > N_2, \abs{u_n - l} < \epsilon
  \]
  Let $N' = \max(N_1, N_2)$. Then $\forall n > N'$,
  \begin{align*}
    \abs{l_n - l} < \epsilon &\iff -\epsilon < l_n - l < \epsilon \\
    &\iff l - \epsilon < l_n < l + \epsilon
  \end{align*}
  and similarly, $l - \epsilon < u_n < l + \epsilon$.

  Assume that
  \[
    \exists N \in \N : \forall n \geq N, l_n \leq a_n \leq u_n
  \]
  Then $\forall n > \max(N, N')$,
  \begin{align*}
    l - \epsilon < l_n \leq a_n \leq u_n < l + \epsilon &\iff l - \epsilon < a_n < l + \epsilon \\
    &\iff -\epsilon < a_n - l < \epsilon \\
    &\iff \abs{a_n - l} < \epsilon
  \end{align*}
\end{proof}
Note that it is possible to use a constant sequence for one of $l_n$ or $u_n$.
\begin{eg}
  (MMT 1 6(b)) Use the sandwich theorem to show that
  \[
    \lim_{n \to \infty} \frac{n!}{n ^ n} = 0
  \]
\end{eg}
\begin{solution}
  Let $a_n = \frac{n!}{n ^ n}$.
  Let $l_n = 0$. Then $\lim_{n \to \infty} l_n = 0$.
  \[
    \forall n > 0, n! > 0 \text{ and } n ^ n > 0 \implies a_n > 0 \implies a_n > l_n
  \]
  Let $u_n = \frac{1}{n}$. Then $\lim_{n \to \infty} u_n = 0$. $\forall n > 0$,
  \begin{align*}
    a_n &= \frac{n!}{n ^ n} \\
    &= \frac{n}{n} \cdot \frac{n - 1}{n} \cdot \frac{n - 2}{n} \cdot ... \cdot \frac{1}{n} \\
    &= 1 \cdot (1 - \frac{1}{n}) \cdot (1 - \frac{2}{n}) \cdot ... \cdot \frac{1}{n}
  \end{align*}
  Since every term $\leq 1$, we have $a_n \leq \frac{1}{n} = u_n$.
  So $\forall n > 0, l_n \leq a_n \leq u_n$.
  Since $l_n \to 0$ and $u_n \to 0$ as $n \to \infty$, by sandwich theorem, $a_n \to 0$.
\end{solution}

% ==================================================================================================

\section{Triangle inequality}
\begin{lemma}
  \[
    \forall x, y > 0, \\
    x \leq y \iff x ^ 2 \leq y ^ 2
    \]
\end{lemma}
\begin{proof}
  Assume $x \leq y$. Then
  \begin{align*}
    x \leq y \implies x ^ 2 &\leq y * x && \text{since $x > 0$} \\ 
    &\leq y * y && \text{since $x \leq y$} \\ 
    &= y ^ 2
  \end{align*}
  So $x \leq y \implies x ^ 2 \leq y ^ 2$.

  Assume $x ^ 2 \leq y ^ 2$. Then
  \begin{align*}
    x ^ 2 \leq y ^ 2 &\implies x ^ 2 - y ^ 2 \leq 0 \\
    &\implies (x - y)(x + y) \leq 0 \\ 
    &\implies x - y \leq 0 && \text{since $x, y > 0 \implies x + y > 0$} \\
    &\implies x \leq y
  \end{align*}
  So $x ^ 2 \leq y ^ 2 \implies x \leq y$.
\end{proof}
\begin{lemma}[Triangle inequality]
  \[
    \abs{a + b} \leq \abs{a} + \abs{b}
  \]
\end{lemma}
\begin{proof}
  \begin{align*}
    \abs{a + b} ^ 2 &= (a + b) ^ 2 \\
    &= a ^ 2 + 2ab + b ^ 2 \\
    &\leq a ^ 2 + 2\abs{ab} + b ^ 2 \\
    &= \abs{a} ^ 2 + 2\abs{ab} + \abs{b} ^ 2 \\
    &= \abs{a} ^ 2 + 2\abs{a}\abs{b} + \abs{b} ^ 2 \\
    &= (\abs{a} + \abs{b}) ^ 2
  \end{align*}
  From lemma, $\abs{a + b} \leq \abs{a} + \abs{b}$.
\end{proof}
Here is another proof of the triangle inequality:
\begin{proof}
  \begin{align*}
    a + b \leq \abs{a} + b \leq \abs{a} + \abs{b} \\
    - a - b \leq \abs{a} - b \leq \abs{a} + \abs{b} \implies a + b \leq -(\abs{a} + \abs{b})
  \end{align*}
  So we have
  \[
    -(\abs{a} + \abs{b}) \leq a + b \leq \abs{a} + \abs{b} \iff \abs{a + b} \leq \abs{a} + \abs{b}
  \]
\end{proof}

% ==================================================================================================

\section{Bounds}
Suppose $X \subseteq \R$.
\begin{itemize}
  \item $u \in \R$ is an upper bound of $X$ if $x \leq u$ for all $x \in X$.
  \item $s \in \R$ is the supremum (least upper bound) of $X$ if $s \leq u$ for all upper bounds $u$ of $X$.
  \item $l \in \R$ is a lower bound of $X$ if $x \geq l$ for all $x \in X$.
  \item $i \in \R$ is the infimum (greatest lower bound) of $X$ if all $i \geq l$ for all lower bounds $l$ of $X$.
\end{itemize}

To describe the bounded conditions of $X$, we have
\begin{itemize}
  \item $X$ is bounded above if $X$ has an upper bound.
  \item $X$ is bounded below if $X$ has a lower bound.
  \item $X$ is bounded if $X$ has an upper bound and a lower bound.
\end{itemize}

% ==================================================================================================

\section{Cauchy sequences}
\begin{definition}
  A sequence $(a_n)_{n \geq 1}$ is a Cauchy sequence (in the real numbers) if and only if
  \[
    \forall \epsilon > 0, \; \exists N \in \N: \forall n, m \in \N, \; n, m > N \implies \abs{a_n - a_m} < \epsilon
  \]
\end{definition}
We will use \textit{convergent} to mean "converging to some limit $l \in \R$".
\begin{prop}
  \label{prop:convergent-cauchy}
  All convergent sequences are Cauchy.
\end{prop}
\begin{proof}
  Suppose the sequence $a_n$ converges to $l \in \R$. Then 
  \[
    \forall \epsilon > 0, \; \exists N \in \N: \forall n \in \N, \; n > N \implies \abs{a_n - l} < \frac{\epsilon}{2}
  \]
  Take arbitrary $\epsilon > 0$. Let $N$ satisfy the above for this $\epsilon$. Take arbitrary $n, m > N$. Then $\abs{a_n - l} < \frac{\epsilon}{2}$ and $\abs{a_m - l} < \frac{\epsilon}{2}$.
  \begin{align*}
    \abs{a_n - a_m} &= \abs{a_n - l - (a_m - l)} \\
    &\leq \abs{a_n - l} + \abs{-(a_m - l)} \ \text{by triangle inequality} \\
    &= \abs{a_n - l} + \abs{a_m - l} \\
    &< \frac{\epsilon}{2} + \frac{\epsilon}{2} \\
    &= \epsilon
  \end{align*}
  Since $n, m, \epsilon$ arbitrary, $a_n$ is a Cauchy sequence.
\end{proof}
Using the contrapositive of this proposition, we present another explanation to why $(-1)^n$ does not converge to a limit $l \in \R$.
\begin{explanation}
  We want to show that $(-1)^n$ is not a Cauchy sequence.
  Observe that
  \[
    \forall N \in \N, \exists n, m > N: \abs{a_n - a_m} = 2
  \]
  Pick some $0 < \epsilon < 2$. Then there does not exist $N \in \N$ such that $\forall n, m > N, \abs{a_n - a_m} < \epsilon$. So $(-1)^n$ is not a Cauchy sequence. By contrapositive, $(-1)^n$ does not converge to a limit $l \in \R$.
\end{explanation}
\begin{prop}
  \label{prop:cauchy-bounded}
  All Cauchy sequences are bounded.
\end{prop}
\begin{proof}
  We want to show that
  \[
    \exists M \in \R: \abs{a_n} \leq M \ \forall n \in \N
  \]
  \begin{remark}
    We are taking the absolute value of $a_n$ purely for convenience; equally, we could have found separate upper and lower bounds, instead of just one $M$, but $\abs{x_n} < M$ encapsulates both bounds perfectly fine. We simply take the magnitude of the bound which has the larger magnitude, then the interval $[-M, M]$ covers both bounds (picture the number line), regardless if they are positive or negative.
  \end{remark}
  Since $a_n$ is a Cauchy sequence, we have
  \[
    \forall \epsilon > 0, \; \exists N \in \N: \forall n, m \in \N, \; n, m > N \implies \abs{a_n - a_m} < \epsilon
  \]
  Let $\epsilon = 1$. Then substituting into the definition above,
  \[
    \exists N \in \N: \forall n, m \in \N, \; n, m > N \implies \abs{a_n - a_m} < 1
  \]
  Let
  \[
    M = 1 + \max\{\abs{a_1}, \abs{a_2}, \abs{a_3}, ..., \abs{a_{N + 1}}\}
  \] 
  \begin{remark}
    Instead of setting $M = \epsilon + ...$, we choose the specific case when $\epsilon = 1$ (or any other value) so that $M$ would not depend on $\epsilon$, which we would want to be arbitrarily small. Note that we are using the $N$ when $\epsilon = 1$, but not any $N$ that satisfies any arbitrary $\epsilon$, so none of the proof depends on the value of $\epsilon$. We can choose any $\epsilon$ and any corresponding $N$, but once we have chosen a value, we fix it regardless of the value of $\epsilon$. This justifies the step of substituting $\abs{a_n - a_{N + 1}} < 1$ in the next part.
  \end{remark}
  Then $\forall n > N$,
  \begin{align*}
    \abs{a_n} &= \abs{a_n + a_{N + 1} - a_{N + 1}} \\
    &\leq \abs{a_n - a_{N + 1}} + \abs{a_{N + 1}} \ \text{by triangle inequality} \\ 
    &< 1 + \abs{a_{N + 1}}\\
    &\leq M
  \end{align*}
  \begin{remark}
    We include $a_{N + 1}$ when defining $M$ because otherwise, we wouldn't know what $\abs{a_n - a_N}$ evaluates to, since the definition for Cauchy only applies for $n > N$ but not $n = N$. $\abs{a_n - a_N}$ isn't directly related to $M$, but we want to be able to do the trick of subtracting and adding the same thing, which is $a_{N + 1}$ in this case.
  \end{remark}
  So $M$ is an upper bound for $a_n$.
\end{proof}
Although we can combine \Cref{prop:convergent-cauchy} and \Cref{prop:cauchy-bounded} to prove the following proposition transitively, we can do so directly in a fairly straightforward manner:
\begin{prop}
  \label{prop:convergent-bounded}
  All convergent sequences are bounded.
\end{prop}
\begin{proof}
  Suppose the sequence $a_n$ converges to some limit $l \in \R$. Then
  \[
    \forall \epsilon > 0, \; \exists N \in \N: \forall n \in \N, \; n > N \implies \abs{a_n - l} < \epsilon
  \]
  Let $\epsilon = 1$ and let $N$ satisfy the above definition when $\epsilon = 1$. Then for all $n > N$, we have
  \begin{align*}
    \abs{a_n} - \abs{l} &\leq \abs{a_n - l} \ \text{by reverse triangle inequality} \\ 
    &< 1 \\
    \iff \abs{a_n} &< \abs{l} + 1
  \end{align*}
  Let
  \[
    M = \max\{\abs{l} + 1, \abs{a_1}, \abs{a_2}, ..., \abs{a_N}\}
  \]
  Then $\forall n > N$, $\abs{a_n} < \abs{l} + 1 \leq M$. Otherwise, $\abs{a_1}, \abs{a_2}, ..., \abs{a_N} \leq M$ by construction of $M$. So $\abs{a_n} \leq M \ \forall n \in \N$.
\end{proof}

% --------------------------------------------------------------------------------------------------

\subsection{Subsequences}
\begin{definition}
  A subsequence is an infinite ordered subset of a sequence.
\end{definition}
Note that a subsequence, by default, has infinitely many terms. Since it is defined to be a subset of a sequence, the sequence from which the subsequence is taken from must also have infinitely many terms. 
\begin{notation}
  We denote a subsequence of a sequence $(a_n)_{n \geq 1}$ as $(a_{n_i})_{i \geq 1}$, where $n_i$ is a strictly increasing sequence in $\N^+$.
\end{notation}
\begin{prop}
  For any subsequence $(a_{n_i})_{i \geq 1}$, $n_i \geq i$.
\end{prop}
\begin{explanation}
A sequence is a subsequence of itself, so in this case we have $n_i = i$ since we are not omitting any terms to form the subsequence (which is the sequence itself). For any other subsequence, suppose the first omitted term is $a_k$. Then 
\[
  \begin{cases}
    n_i = i & i < k \\ 
    n_i > i & i \geq k
  \end{cases}
\]  
so in general, for any subsequence, we have $n_i \geq i$.
\end{explanation}
\begin{theorem}
  Let $a_n$ be a sequence and $a_{n_i}$ be a subsequence of $a_n$. If $a_n$ converges to limit $l \in \R$, then $a_{n_i}$ also converges to $l$.
\end{theorem}
\begin{proof}
  Since $a_n$ converges to $l$, given some $\epsilon > 0$, 
  \[
    \exists N \in \N: \forall i > N, \abs{a_i - l} < \epsilon
  \]
  Since $n_i \geq i$, $i > N$ implies $n_i > N$, so
  \[
    \forall i > N, \abs{a_{n_i} - l} < \epsilon
  \]
\end{proof}
\begin{theorem}
  Every subsequence of real numbers has a monotonic subsequence.
\end{theorem}
\begin{proof}
  Let $a_n$ be a peak if
  \begin{align*}
    a_n > a_m && \forall m > n
  \end{align*}
  If the sequence has infinitely many peaks at $n_1 < n_2 < n_3 < ...$, then $a_{n_1} > a_{n_2} > a_{n_3} > ...$ is a strictly decreasing subsequence.
  
  Otherwise, the sequence has only finitely many peaks at $n_1 < n_2 < ... < n_k$, or no peaks. Consider $a_{n_k + 1}$. Since $a_{n_{k + 1}}$ is not a peak,
  \[
    \exists i > k + 1: a_{n_i} \geq a_{n_k + 1}
  \]
  and since $a_{n_i}$ is not a peak, we can construct a (not strictly) increasing subsequence in this way.
\end{proof}

% --------------------------------------------------------------------------------------------------

\subsection{Completeness}
We define two notions of completeness:
\begin{definition}
  A metric space $(X, d)$ is Cauchy complete if and only if all Cauchy sequences in $(X, d)$ converges to an element in $X$.
\end{definition}
\begin{eg}
  The set of rational numbers $\Q$ is not Cauchy complete. 
  
  Consider the sequence
  \[
    a_1 = 1, a_{n + 1} = \frac{x_n + \frac{2}{x_n}}{2}
  \]
  Every term in the sequence is in $\Q$, but the sequence converges to $\sqrt{2}$, which is not in $\Q$.

  Consider another sequence in the rationals
  \[
    a_1 = 3.1, a_2 = 3.14, a_3 = 3.142, a_4 = 3.1416, ...
  \]
  where $a_n$ represents the $n$th decimal approximation of $\pi$. The sequence converges to $\pi \notin \Q$.
\end{eg}
\begin{definition}[Axiom of Dedekind completeness]
  A partially ordered set $X$ is Dedekind complete if and only if it has the least-upper-bound property, i.e. every non-empty subset of $X$ with an upper bound has a least upper bound (i.e. supremum).
\end{definition}
\begin{theorem}[Fundamental theorem of analysis]
  Every increasing sequence of real numbers $a_n$ with an upper bound:
  \begin{itemize}
    \item has a supremum $s$, and
    \item converges to $s$.
  \end{itemize}
\end{theorem}
\begin{proof}
  A non-empty subset of the partially ordered set $\R$ is equivalent to an increasing sequence of real numbers. So, the axiom of Dedekind completeness tells us that every increasing sequence of real numbers with an upper bound has a supremum.

  We prove the second point in two stages. 

  In the first stage, we claim that
  \[
    \exists N \in \N (\abs{a_N - s} < \epsilon)
  \]
  We prove this by contradiction. Suppose $\forall n \in \N, \abs{a_n - s} \geq \epsilon$. Since $s$ is the supremum, we have $s \geq a_n \forall n \in \N$, so $s - a_n \geq \epsilon \forall n \in \N$. Rearranging the inequality gives
  \begin{align*}
    a_n \leq s - \epsilon && \forall n \in \N
  \end{align*}
  By definition, $s - \epsilon$ is an upper bound. However, since $\epsilon > 0$, $s - \epsilon < s$, which contradicts the fact that $s$ is the supremum, i.e. least upper bound. So the claim is true.

  In the second stage, we claim that
  \[
    \forall n \in \N (n > N \implies \abs{a_n - s} < \epsilon)
  \]
  where $N$ satisfies the criterion in our first claim. Since $a_n$ increasing, $n > N \implies a_n \geq a_N$. So
  \begin{align*}
    s - a_n &\leq s - a_N \\
    &< \epsilon && \text{from above}
  \end{align*}
  Since $s - a_n \geq 0$, we have $\abs{a_n - s} < \epsilon$.
\end{proof}
